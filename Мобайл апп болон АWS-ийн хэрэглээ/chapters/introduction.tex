\section{Товч танилцуулга}
	\quad \quad	ХандПро компани нь "Тогтвортой хөгжлийн зорилго 2030"-н 1, 3, 4, 8, 9, 10, 11 зорилтуудыг дэмжиж ажилладаг. Уг компани нь хөгжлийн бэрхшээлтэй иргэдийн нийгмийн харилцаанд тулгардаг маш олон асуудлуудыг суурь судалгаанд үндэслэн хиймэл оюун ухаан болон технологид суурилан шийдэхийг эрмэлздэг. Ингэхдээ тус шийдлээ үе шаттайгаар турших, хэрэглээнд нэвтрүүлэх улмаар хөгжлийн бэрхшээлтэй хүмүүсийг нийгмийн харилцаанд идэвхтэй оролцдог болгох, хөдөлмөр эрхлэлтийн түвшнийг дээшлүүлэх, боловсрол болон амьдралын чанарыг нэмэгдүүлэх зорилго бүхий ногоон технологийн компани юм. Баг хамт олны хувьд Япон, Солонгос, Америк, Монгол гэсэн 4 орноос нэгдсэн 5 - 20 жилийн хиймэл оюун, үүлэн технологи, мобайл, вэб, код эксперт зэрэг технологийн туршлагатай баг мөн 10-с дээш жилийн туршлагатай ддохионы хэлний хэлмэрч, бусад хөржлийн бэрхшээлтэй 2 ажилтан зэрэг нийт 14 хүний бүрэлдэхүүнтэй Diverse хамт олон юм.

\section{Эрхэм зорилго}
        \quad\quad Технологид суурилсан хөгжлийн шийдэл, нийгмийн эрх тэгш оролцоо, ялгаварлан гадуурхалтгүй нийгмийг бид бүтээлцэнэ.

\section{Ямар үйлчилгээ үзүүлдэг вэ?}
	\quad \quad	Технологи дээр суурилж хөгжлийн бэрхшээлтэй иргэдэд эрх тэгш орчинг бүрдүүлэх технологийн бүтээгдэхүүн хөгжүүлдэг компани юм. Хөгжүүлэгдэж буй төслүүдээс дурдвал,
     \begin{enumerate}
      \item Виртуал дохионы хэлмэрч
      \item Tuslamj platform
      \item E-Hurteemj
      \item Hand Pro assistant
      \item Demney marketplace platform
      \item AVR+
      \item Disability expo
      \item Hurteemj social content
    \end{enumerate}

 \section{Программ хангамж}
        \quad \quad Программ хангамжийн хувьд front-end хөгжүүлэлтэд Javascript, Next.js, React дээр, back-end хөгжүүлэлтэд AWS технологи, мобайл апп хөгжүүлэлтэд Flutter фреймворк ашигладаг. 

